\subsection{Терминология}
\begin{description}


\item[Блок трансляции]
Множество базовых блоков подлежащие трансляции в коды команд основной системы.

\item[Базовый блок, англ. basic block]

Максимальная последовательность следующих друг за другом команд, обладающих следующими свойствами: 1) поток управления может входить в базовый блок только через первую команду блока. 2) управление покидает блок без останова или ветвления, за исключением возможно в последней команде блока.

\item[Граф потока, англ. flow graph]

Граф узлами которого являются базовые блоки, а ребра которого указывают порядок следования блоков.

\item[Распределение регистров, англ. register allocation]

Задача определения множества переменных, которые будут находится в регистрах в каждой точке программы.

\item[Назначение регистров, англ. register assignement]

Задача выбора конкретных регистров для размещения в них переменных.

\item[Сохранение или сброс регистра, англ. register spilling]

Сохранение (сброс - spilled) содержимого регистра в ячейку памяти для освобождения регистра. Необходимо
когда для вычисления требуется регистр, а все доступные регистры уже используются.



\end{description}

