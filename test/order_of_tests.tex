%=========================================
\subsection{Параметры технических средств, используемых во время испытаний}
Для работы алгоритма в эмуляторе QEMU необходимо учесть следующие системные требования:
\begin{my_enumerate}
\item Компьютер, оснащенный:
    \begin{my_enumerate}
    \item 64-разрядный (x86\_64) процессор с тактовой частотой 1 гигагерц (ГГц) или выше;
    \item 2 ГБ оперативной памяти (ОЗУ);
    \item 1.5 ГБ свободного места на жестком диске;
    \end{my_enumerate}
\item Монитор
\item Мышь
\item Клавиатура
\end{my_enumerate}


%=========================================
\subsection{Программные средства, необходимые для проведения испытаний}
Реализация алгоритма для распределения глобальных регистров в эмуляторе QEMU обязательно должен быть написан с использованием языка C. На компьютере должны быть установленны программы необходимые для сборки проекта QEMU. Алгоритм должен включаться в сборку эмулятора QEMU и работать на 64-разрядных процессорах под операционной системой Linux.


%=========================================
\subsection{Порядок проведения испытаний}
Испытания должны проводиться в следующем порядке:
\begin{my_enumerate}
\item Проверка требований к документации.
\item Проверка требований к интерфейсу.
\item Проверка требований к функциональным возможностям программы.
\item Проверка требований надежности.
\end{my_enumerate}


%=========================================
\subsection{Условия проведения испытаний}

\subsubsection{Требования к численности и калификации персонала}
Минимальное количество персонала, требуемого для работы программы: 1 оператор. Пользователь программы должен иметь образование не ниже среднего, обладать практическими навыками работы с компьютером и электронникой.


