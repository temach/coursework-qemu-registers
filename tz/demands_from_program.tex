

%=========================================
\subsection{Требования к функциональным характеристикам}
\subsubsection{Состав выполняемых функций}
\begin{my_enumerate}
\item Интеграция в QEMU версии 2.10 или позднее.
\item Реализация алгоритма распределения регистров внутри блока трансляции.
\end{my_enumerate}

\subsubsection{Организация входных и выходных данных}
Входными данными для работы алгоритма является массив инструкций для блока трансляции в формате внутреннего представления эмулятора QEMU. Для работы алгоритма необходима исполняемая программа, которая может быть запущена в эмуляторе QEMU. Входной файл исполняемой программы может быть создан в любой среде разработки на платформе которую поддерживает эмулятор QEMU, например х86\_64 с операционной системой Linux.

\begin{my_enumerate}
\item Файл программы должен представлять собой исполняемый файл предназначенный для запуска в userspace операционной системы Linux на архитектуре х86\_64.
\item Файл программы должен быть предоставлен в формате ELF.
\end{my_enumerate}

\medskip
Выходными данными для алгоритма являются коды команд для архитектуры х86\_64.

\subsubsection{Прочие требования}
\begin{enumerate}
\item Отсутствие влияния на корректное исполнение программы.
\end{enumerate}


%=========================================
\subsection{Требования к интерфейсу}
Требования к интерфейсу не предъявляются.

%=========================================
\subsection{Требования к надежности}
\subsubsection{Обеспечение устойчивого функционирования программы}
При некорректных входных параметрах должно отображаться сообщение об ошибке.
\subsubsection{Время восстановления после отказа}
Требования к восстановлению после отказа не предъявляются.
\subsubsection{Отказы из-за некорректных действий оператора}
Требования к отказу из-за некорректных действий оператора не предъявляются.

%=========================================
\subsection{Требования к условиям эксплуатации}
\subsubsection{Вид обслуживания}
Не требует каких-либо видов обслуживания.
\subsubsection{Численность и квалификация персонала}
Минимальное количество персонала, требуемого для работы: 1 оператор. Пользователь эмулятора QEMU должен иметь образование не ниже среднего, обладать практическими навыками работы с компьютером.

%=========================================
\subsection{Требования к составу и параметрам технических средств}
Для работы алгоритма в эмуляторе QEMU необходимо учесть следующие системные требования:
\begin{my_enumerate}
\item Компьютер, оснащенный:
    \begin{my_enumerate}
    \item 64-разрядный (x86\_64) процессор с тактовой частотой 1 гигагерц (ГГц) или выше;
    \item 2 ГБ оперативной памяти (ОЗУ);
    \item 1.5 ГБ свободного места на жестком диске;
    \end{my_enumerate}
\item Монитор
\item Мышь
\item Клавиатура
\end{my_enumerate}


%=========================================
\subsection{Требования к информационной и программной совместимости}

Реализация алгоритма для распределения глобальных регистров в эмуляторе QEMU обязательно должен быть написан с использованием языка C. Алгоритм должен включаться в сборку эмулятора QEMU и работать на 64-разрядных процессорах под операционной системой Linux.


%=========================================
\subsection{Требования к упаковке}
Реализация алгоритма поставляется в виде патча для исходного кода эмулятора QEMU на внешнем носителе информации – USB флеш накопителе. На нем должны содержаться документация по разработке, исходный код для алгоритма.
