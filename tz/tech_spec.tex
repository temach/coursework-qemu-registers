\documentclass[
%a4paper,12pt
encoding=utf8
]{../twoeskd}

% \usepackage{eskdappsheet}

% Packages required by doxygen
\usepackage[export]{adjustbox} % also loads graphicx
\usepackage{graphicx}
\usepackage[utf8]{inputenc}
\usepackage{multicol}
\usepackage{multirow}
\usepackage{makeidx}

% NLS support packages
\usepackage[T2A]{fontenc}
\usepackage[russian]{babel}
\usepackage{pscyr}

% Font selection
\usepackage{courier}
\usepackage{amssymb}

\setlength{\parindent}{0cm}
\setlength{\parskip}{0.2cm}

% debug to see the frame borders
% from https://en.wikibooks.org/wiki/LaTeX/Page_Layout
% \usepackage{showframe}

% Indices & bibliography
\usepackage{natbib}
\usepackage[titles]{tocloft}
\setcounter{tocdepth}{3}
\setcounter{secnumdepth}{5}

% change style of titles in \section{}
\usepackage{titlesec}
\titleformat{\section}[hang]{\huge\bfseries\center}{\thetitle.}{1em}{}
\titleformat{\subsection}[hang]{\Large\normalfont\raggedright}{\thetitle.}{1em}{\underline}
\titleformat{\subsubsection}[hang]{\large\normalfont\raggedright}{\thetitle.}{1pt}{}

% Packages for text layout in normal mode
% \usepackage[parfill]{parskip} % автоматом делает пустые линии между параграфами, там где они есть в тексте
% \usepackage{indentfirst} % indent even in first paragraph
\usepackage{setspace}	 % controls space between lines
\setstretch{1} % space between lines
\setlength\parindent{0.9cm} % size of indent for every paragraph
\usepackage{csquotes}% превратить " " в красивые двойные кавычки
\MakeOuterQuote{"}


% this makes items spacing single-spaced in enumerations.
\newenvironment{my_enumerate}{
\begin{enumerate}
  \setlength{\itemsep}{1pt}
  \setlength{\parskip}{0pt}
  \setlength{\parsep}{0pt}}{\end{enumerate}
}


% Custom commands
% configure eskd
\titleTop{
\textbf{\Large ПРАВИТЕЛЬСТВО РОССИЙСКОЙ ФЕДЕРАЦИИ \\
НАЦИОНАЛЬНЫЙ ИССЛЕДОВАТЕЛЬСКИЙ УНИВЕРСИТЕТ \\
«ВЫСШАЯ ШКОЛА ЭКОНОМИКИ» } \\
\vspace*{0.2cm}
{\small Факультет компьютерных наук \\
Департамент программнoй инженерии \\
}
}
\titleDesignedBy{Студент группы БПИ 151 НИУ ВШЭ}{Абрамов А.M.}
\titleAgreedBy{%
\parbox[t]{7cm} {
Профессор базовой кафедры \\
системного программирования \\
В НИУ ВШЭ \\
канд. физ. - мат. наук \\
}}{Гайсарян С. С.}
\titleApprovedBy{
\parbox[t]{10cm} {
Академический руководитель \\
образовательной программы \\
«Программная инженерия» \\
профессор департамента программной \\
инженерии канд. техн. наук \\
}}{Шилов В. В.}
\titleName{АЛГОРИТМ ДЛЯ ГЛОБАЛЬНОГО РАСПРЕДЕЛЕНИЯ РЕГИСТРОВ В ЭМУЛЯТОРЕ QEMU И ЕГО РЕАЛИЗАЦИЯ}
\workTypeId{RU.17701729.509000 T3 01-1}

\titleSubname{Техническое задание}


%===== C O N T E N T S =====


\begin{document}

% Titlepage & ToC
\pagenumbering{roman}

% some water filling text, that is pointless but adds text
% \input{annotation}

\newpage
\pagenumbering{arabic}
\tableofcontents

% --- add my custom headers ---
\newpage
\section{Введение}
\subsection{Наименование}
Наименование: «Алгоритм для глобального распределения регистров в эмуляторе QEMU и его реализация». \\
Наименование на английском: «Algorithm for global management of registers in the QEMU emulator and its implementation». \\

\subsection{Краткая характеристика}
    Цель работы - составить и реализовать алгоритм для глобального распределения регистров в эмуляторе QEMU.
    В задачи работы вошло рассмотрение уже существующих алгоритмов, разработка алгоритма и его реализация.
    Рассмотрение уже существующих алгоритмов для глобального распределения регистров позволило выявить их характеристики. Основываясь на анализе разработан алгоритм для глобального распределения регистров.
    В состав работы также вошло создание демонстрационных исходных данных (файлов) для проверки работы алгоритма. Входной для эмулятора файл программы в формате ELF, удовлетворяющий требованиям входных данных, может быть получен в результате компиляции исходного кода одним из компиляторов, например gcc или llvm.

\subsection{Основание для разработки}
Разработка программы ведется на основании приказа декана факультета компьютерных наук
Национального исследовательского университета «Высшая школа экономики» 
\textnumero 2.3-02/1112-01 от 12.12.2017
«Об  утверждении  тем,  руководителей  курсовых  работ  студентов
образовательной  программы  Программная  инженерия 
факультета 
компьютерных наук».

\newpage
\section{Основания для разработки}
\subsection{Документ, на основании которого ведется разработка}
Разработка программы ведется на основании приказа декана факультета компьютерных наук  
Национального исследовательского университета «Высшая школа экономики» 
\textnumero 2.3-02/1212-01 от 12.12.17
«Об  утверждении  тем,  руководителей  курсовых  работ  студентов
образовательной  программы  Программная  инженерия 
факультета 
компьютерных наук».


\subsection{Наименование темы разработки}
Наименование: «Алгоритм для глобального распределения регистров в эмуляторе QEMU и его реализация». \\
Наименование на английском: «Algorithm for global allocation of registers in the QEMU emulator and its implementation». \\


\newpage
\section{Назначение разработки}
\subsection{Функциональное назначение}
Функциональным назначением разработки является предоставление пользователю возможности ускорить работу эмулятора QEMU.


\subsection{Эскплутационное назначение}
Реализованный алгоритм предназначен для включения в сборку программы QEMU на операционной системе Linux. Алгоритм может использоватся любым пользователем желающем ускорить работу эмулятора QEMU. Исходный код может использоваться в учебных целях как пример реализации алгоритма тесно взаимодействующего с внутренними механизмами QEMU.

\newpage
\section{Требования к программному изделию}


%=========================================
\subsection{Требования к функциональным характеристикам}
\subsubsection{Состав выполняемых функций}
\begin{my_enumerate}
\item Интеграция в QEMU версии 2.10 или позднее.
\item Реализация алгоритма распределения регистров внутри блока трансляции.
\end{my_enumerate}

\subsubsection{Организация входных и выходных данных}
Входными данными для работы алгоритма является массив инструкций для блока трансляции в формате внутреннего представления эмулятора QEMU. Для работы алгоритма необходима исполняемая программа, которая может быть запущена в эмуляторе QEMU. Входной файл исполняемой программы может быть создан в любой среде разработки на платформе которую поддерживает эмулятор QEMU, например х86\_64 с операционной системой Linux.

\begin{my_enumerate}
\item Файл программы должен представлять собой исполняемый файл предназначенный для запуска в userspace операционной системы Linux на архитектуре х86\_64.
\item Файл программы должен быть предоставлен в формате ELF.
\end{my_enumerate}

\medskip
Выходными данными для алгоритма являются коды команд для архитектуры х86\_64.

\subsubsection{Прочие требования}
\begin{enumerate}
\item Отсутствие влияния на корректное исполнение программы.
\end{enumerate}


%=========================================
\subsection{Требования к интерфейсу}
Требования к интерфейсу не предъявляются.

%=========================================
\subsection{Требования к надежности}
\subsubsection{Обеспечение устойчивого функционирования программы}
При некорректных входных параметрах должно отображаться сообщение об ошибке.
\subsubsection{Время восстановления после отказа}
Требования к восстановлению после отказа не предъявляются.
\subsubsection{Отказы из-за некорректных действий оператора}
Требования к отказу из-за некорректных действий оператора не предъявляются.

%=========================================
\subsection{Требования к условиям эксплуатации}
\subsubsection{Вид обслуживания}
Не требует каких-либо видов обслуживания.
\subsubsection{Численность и квалификация персонала}
Минимальное количество персонала, требуемого для работы: 1 оператор. Пользователь эмулятора QEMU должен иметь образование не ниже среднего, обладать практическими навыками работы с компьютером.

%=========================================
\subsection{Требования к составу и параметрам технических средств}
Для работы алгоритма в эмуляторе QEMU необходимо учесть следующие системные требования:
\begin{my_enumerate}
\item Компьютер, оснащенный:
    \begin{my_enumerate}
    \item 64-разрядный (x86\_64) процессор с тактовой частотой 1 гигагерц (ГГц) или выше;
    \item 2 ГБ оперативной памяти (ОЗУ);
    \item 1.5 ГБ свободного места на жестком диске;
    \end{my_enumerate}
\item Монитор
\item Мышь
\item Клавиатура
\end{my_enumerate}


%=========================================
\subsection{Требования к информационной и программной совместимости}

Реализация алгоритма для распределения глобальных регистров в эмуляторе QEMU обязательно должен быть написан с использованием языка C. Алгоритм должен включаться в сборку эмулятора QEMU и работать на 64-разрядных процессорах под операционной системой Linux.


%=========================================
\subsection{Требования к упаковке}
Реализация алгоритма поставляется в виде патча для исходного кода эмулятора QEMU на внешнем носителе информации – USB флеш накопителе. На нем должны содержаться документация по разработке, исходный код для алгоритма.


\newpage
\section{Требования к программной документации}
\subsection{Предварительный состав программной документации}
В обязательном порядке должны входить:
\begin{my_enumerate}
\item Техническое задание  (ГОСТ 19.201-78)
\item Пояснительная записка  (ГОСТ 19.404-79)
\item Руководство оператора  (ГОСТ 19.505-79)
\item Программа и методика испытаний (ГОСТ 19.301-79*)
\item Текст программы  (ГОСТ 19.401-78*)
\end{my_enumerate}



\newpage
\section{Технико-экономические показатели}
\subsection{Ориeнтировочная экономическая эффективность}
Ориeнтировочная экономическая эффективность не рассчитывается.

\subsection{Экономические преимущества разработки}
Ориeнтировочны экономические преимущества разработки не рассчитывается.

\newpage
\section{Стадии и этапы разработки}

%=========================================
\subsection{Необходимые стадии разработки}

\subsubsection{Стадия разработки технического задания:}
\begin{my_enumerate}
\item Этап обоснования необходимости разработки программы:
    \begin{my_enumerate}
    \item постановка задачи.
    \item сбор исходных материалов.
    \end{my_enumerate}
\item Этап разработки и утверждения технического задания:
    \begin{my_enumerate}
    \item определение требований к алгоритму.
    \item определение стадий, этапов и сроков разработки программы и документации на нее.
    \item согласование и утверждение технического задания.
    \end{my_enumerate}
\end{my_enumerate}

\subsubsection{Стадия разработки технического проекта:}
\begin{my_enumerate}
\item Этап исследования уже существующих решений:
    \begin{my_enumerate}
    \item поиск уже созданных решений
    \item изучение их структуры и архитектуры
    \item анализ их рабочих характеристик
    \end{my_enumerate}
\item Этап разработки технического проекта:
    \begin{my_enumerate}
    \item разработка алгоритма
    \item разработка структуры и архитектуры частей алгоритма.
    \item анализ оптимальности найденного алгоритма
    \end{my_enumerate}
\item Этап утверждения технического проекта:
    \begin{my_enumerate}
    \item разработка плана мероприятий по разработке программы
    \item разработка пояснительной записки.
    \end{my_enumerate}
\end{my_enumerate}


\subsubsection{Стадия разработки рабочего проекта:}
\begin{my_enumerate}
\item Этап разработки программы:
    \begin{my_enumerate}
    \item непосредственное программирование и отладка алгориттма.
    \end{my_enumerate}
\item Этап разработки программной документации:
    \begin{my_enumerate}
    \item разработка следующих программных документов в соответствии с требованиями: техническое задание, пояснительная записка, руководство оператора, программа и методика испытания, текст программы, все в соответствии с требованиями ГОСТ 19.101-77.
    \end{my_enumerate}
\item Этап испытания программы:    
    \begin{my_enumerate}
    \item разработка, согласование и утверждение программы и методики испытаний.
    \item испытания программы.
    \item защита презентации, сдача разработанной документации.
    \item корректировка программы и программной документации по результатам испытаний.
    \end{my_enumerate}
\end{my_enumerate}


%=========================================
\subsection{Сроки работ и исполнители}
% TODO change date to the real date
Алгорим должен быть разработан к 10 мая 2018 года, студентом группы БПИ151 Абрамовым Артемом.


\newpage
\section{Порядок контроля и приемки}
\subsection{Виды испытаний}
Контроль и приемка разработки осуществляются в соответствии с разработанным исполнителем и согласованным с заказчиком документом «Алгоритм для глобального распределения регистров в эмуляторе QEMU и его реализация» Программа и методика испытаний по (ГОСТ 19.301-79*).

\subsection{Требования к приемке работы}
Акт приемки-сдачи программы между исполнителем и заказчиком в эксплуатацию происходит при полной работоспособности алгоритма, при выполнении указанных в настоящем документе функций и требований, при наличии документации к программе, выполненной в соответствии с требованиями настоящего технического задания.

\newpage
\section{Приложение 1. Терминология}
\subsection{Терминология}
\begin{description}

\item[Архитектура набора команд (англ. instruction set architecture, ISA)]
часть архитектуры компьютера, определяющая программируемую часть ядра микропроцессора. На этом уровне определяются реализованные в микропроцессоре конкретного типа

\item[контрольный таймер, англ. Watchdog timer]
аппаратно реализованная схема контроля над зависанием системы. Представляет собой таймер, который периодически сбрасывается контролируемой системой. Если сброса не произошло в течение некоторого интервала времени, происходит принудительная перезагрузка системы. В некоторых случаях сторожевой таймер может посылать системе сигнал на перезагрузку («мягкая» перезагрузка), в других же — перезагрузка происходит аппаратно (замыканием сигнального провода RST или подобного ему).


\end{description}



%\newpage
%\section{Приложение 2. Схема интерфейса программы}
%\begin{figure}[h!]
    \centering
    \includegraphics[width=0.8\textwidth]{../screenshots/main_empty.png}
    \caption{Схема интерфейса}
\end{figure}


\newpage
\section{Приложение 2. Список используемой литературы}
\subsection{ Список используемой литературы}
\begin{my_enumerate}
\item
Bellard Fabrice. QEMU, a Fast and Portable Dynamic Translator \\
Proceedings of the Annual Conference on USENIX Annual Techinal Conference. 2005. \\

\item
Smith J., Nair R. Virtual Machines: Versatile Platrofms for Systems and Processes \\
500 Sansome Strees, Suite 400, San Francisco Morgan, CA 94111: Elsevier Inc., 2005 \\

\item Quality and Speed in Linear-scan Register Allocation \\
Omri Traub, Glenn Holloway, Michael D. Smith \\
Harvard University, Division of Engineering and Applied Sciences Cambridge, MA 02138 \\

\item REGISTER ALLOCATION \& SPILLING VIA GRAPH COLORING \\
G. J. Chaitin \\
IBM Research, P.O.Box 218, Yorktown Heights, NY 10598 \\

\item Linear Scan Register Allocation \\
MASSIMILIANO POLETTO, Laboratory for Computer Science, MIT \\
VIVEK SARKAR IBM Thomas J. Watson Research Center \\


\end{my_enumerate}



% Index
\newpage
\eskdListOfChanges

% \phantomsection
% \addcontentsline{toc}{section}{Алфавитный указатель}
% \printindex

\end{document}
