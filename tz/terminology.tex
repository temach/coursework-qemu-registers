\subsection{Терминология}
\begin{description}

\item[Архитектура набора команд (англ. instruction set architecture, ISA)]
часть архитектуры компьютера, определяющая программируемую часть ядра микропроцессора. На этом уровне определяются реализованные в микропроцессоре конкретного типа

\item[контрольный таймер, англ. Watchdog timer]
аппаратно реализованная схема контроля над зависанием системы. Представляет собой таймер, который периодически сбрасывается контролируемой системой. Если сброса не произошло в течение некоторого интервала времени, происходит принудительная перезагрузка системы. В некоторых случаях сторожевой таймер может посылать системе сигнал на перезагрузку («мягкая» перезагрузка), в других же — перезагрузка происходит аппаратно (замыканием сигнального провода RST или подобного ему).


\end{description}

